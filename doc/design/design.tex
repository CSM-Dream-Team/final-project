\documentclass[titlepage,12pt]{article}
\usepackage[margin=1in]{geometry}
\usepackage[inline]{enumitem}
\usepackage{titlesec}
\usepackage{textcomp}
\usepackage{amsmath}
\usepackage{graphicx}
\usepackage[pdftex]{hyperref}
\usepackage[english]{babel}
\usepackage{csquotes}
\usepackage{titling}
\usepackage{titlesec}
\usepackage{xspace}
\usepackage{tabularx}
\usepackage{float}
\usepackage{parskip}

\MakeOuterQuote{"}

\newcommand\tab[1][.5in]{\hspace*{#1}}

\newcommand\name{VRsh\xspace}

\title{Final Project Design v. 1.0}
\author{Sumner Evans, Robbie Merillat, Sam Sartor}
\date{\today}

\begin{document}

\maketitle

\section{Design History}
\begin{table}[H]
    \caption{Version History}
    \label{tbl:version_history}
    \centering
    \begin{tabularx}{\linewidth}{| l | l || X |}
        \hline
        \textbf{Version Number} & \textbf{Date} & \textbf{Change Description} \\
        \hline\hline
        1.0 & 2017-11-07 & First Draft \\
        \hline
    \end{tabularx}
\end{table}

\section{Overview}
VR Technology has been developing rapidly in the 21st century. Current solutions
such as Unity attempt to use old programming languages and paradigms to
implement VR environments and thus limit developers' abilities to create new and
unique environments. With every cutting edge technology, new paradigms and
design patterns must be invented. This project intends to explore and implement
these paradigms and design patterns.

For this project, we will utilize a 3D space to implement an intuitive UI for
\begin{enumerate*}
\item loading programs,
\item saving program state,
\item and customizing the environment
\end{enumerate*} (described in Section~\ref{sec:design}).  We believe that
implementing a shell-like environment is the most effective method for us to
explore those patterns.

\section{Design}\label{sec:design}
There are three main goals for our project (referred to as \textit{\name} for
now):
\begin{enumerate}
    \item \textbf{Loading Programs:} the user will be able to load our three
        individual programs from our \name as described in Section~\ref{sec:ui}.
    \item \textbf{Saving Program State:} the user will be able to save the state
        of an environment. This is elaborated in Section~\ref{sec:state}.
    \item \textbf{Customizing the Environment:} the user will be able to
        customize the \name environment.
\end{enumerate}

\subsection{Visual Design}
The visual design of this program is dynamic in that both the general 
settings and environment can be modified. This would look and feel 
similar to the Steam VR personal "desktop" with added options for settings
and environment customization. "Running" a program will switch from the \name 
home environment to the desired program environment.

\subsubsection{Global UI}\label{sec:ui}
There will be a persistent user interface that will be visible from the \name
home environment and from within the individual environments that can be
accessed from within \name. This UI will a ``halo'' centered at the user,
located above the user's head. The halo will have components attached to allow
the user to perform actions such as running programs and moving between
environments. We have not yet determined the most intuitive design motif for
these controls. However, we have brainstormed a few innovative ideas for these
controls including pull-down selectors, Rolodex selectors, and system setting
asteroid field of options.

\subsubsection{Customizable Environment}\label{sec:env}

\subsubsection{Interactions}

Interactions will include:
\begin{itemize}
    \item Object surface/node snapping
    \item A yank control to bring objects closer/further
    \item Grabbing an object at a point
    \item Throwing objects (grab maintains momentum)
    \item Expanding fixed menus
    \item Applying tool objects to other objects
    \item Inter-object collisions
\end{itemize}

\subsubsection{Lighting Model} We have several lighting/rendering models
available, however most assets will use physically based rendering.

\subsection{System State Storage}\label{sec:state}

\section{Management}

\subsection{Project Scope}
% A summary of the scope of the software

\subsection{Detailed Schedule}
Table~\ref{tab:milestones} shows the five major milestones and their due dates.
\begin{table}[H]
    \caption{Milestone Delivery Schedule}
    \label{tab:milestones}
    \centering
    \begin{tabular}{|l|l|}
        \hline
        \textbf{Milestone} & \textbf{Date} \\
        \hline\hline
        Milestone I: Design Document & 2017/11/08 \\
        \hline
        Milestone II: Individual Project Program Picking & 2017/11/22 \\
        \hline
        Milestone III: Final Code Submission & 2017/12/13 \\
        \hline
    \end{tabular}
\end{table}

\subsection{Risk Analysis}

\subsection{Test Plan}

\subsection{Demo Plan}
Our goal is to have this project ready for the C-MAPP event on January 18, 2018.

\section{Dependencies}

\end{document}
