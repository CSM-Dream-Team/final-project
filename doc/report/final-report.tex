\documentclass[conference,12pt]{IEEEtran}
\usepackage[utf8]{inputenc}
\usepackage[english]{babel}
\usepackage{csquotes}
\usepackage{biblatex}
\usepackage{amsmath}
\usepackage{minted}

\MakeOuterQuote{"}
\addbibresource{references.bib}

\setminted{autogobble,python3,mathescape,fontsize=\footnotesize}
\newcommand\tab[1][.5in]{\hspace*{#1}}
\newcommand\name{VRsh}

% TODO: better name
\title{Final Project Report}
\author{
    \IEEEauthorblockN{
        Jonathan Sumner Evans\IEEEauthorrefmark{1},
        Robinson Merillat\IEEEauthorrefmark{2} and
        Sam Sartor\IEEEauthorrefmark{3}
    }
    \IEEEauthorblockA{
        Department of Computer Science,
        Colorado School of Mines\\
        Golden, Colorado\\
        Email:
            \IEEEauthorrefmark{1}jonathanevans@mines.edu,
            \IEEEauthorrefmark{2}rdmerillat@mines.edu,
            \IEEEauthorrefmark{3}ssartor@mines.edu
    }
}

\begin{document}

\maketitle

\begin{abstract}
VR Technology has been developing rapidly in the 21st century. Current solutions
such as Unity attempt to use old programming languages and paradigms to
implement VR environments and thus limit developers' abilities to create new and
unique environments. With every cutting edge technology, new paradigms and
design patterns must be invented. In this paper, we discuss a new modular,
immediate mode, promise-based application design philosophy suitable for
implementing large virtual reality applications. We present a library which
implements this design philosophy and a few case studies of this library in use.
\end{abstract}

\section{Introduction}
\subsection{Reasons for Exploring Alternative Application Design Philosophies}
\begin{itemize}
    \item Frame Rate
    \item 3D Environment
\end{itemize}
% TODO: something about our goals

\section{Failed Attempts Leading to Flight}
\subsection{Entity Component System}

\section{Flight}

\subsection{Language}
\subsection{Dependencies}
\subsection{Modular}
\subsection{Immediate Mode}
\subsection{Promise-Based}

\section{Case Studies}
\subsection{Let's Get Physical and Snowflakes --- Physics}
\subsection{Workbench --- SOMETHING} % TODO
\subsection{{\name} --- Global User Interface}

\section{Comparison to Alternative Libraries}
\subsection{A-Frame}

\section{Conclusion}

\printbibliography

\end{document}
