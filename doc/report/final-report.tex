% vim: spell spelllang=en_us
\documentclass[conference,12pt]{IEEEtran}
\usepackage[T1]{fontenc}
\usepackage[utf8]{inputenc}
\usepackage[english]{babel}
\usepackage{csquotes}
\usepackage{biblatex}
\usepackage{amsmath}
\usepackage{minted}

\MakeOuterQuote{"}

\addbibresource{references.bib}

\setminted{autogobble,python3,mathescape,fontsize=\footnotesize}
\newcommand\tab[1][.5in]{\hspace*{#1}}
\newcommand\name{VRsh}

% TODO: better name
\title{Final Project Report}
\author{%
    \IEEEauthorblockN{%
        Jonathan Sumner Evans\IEEEauthorrefmark{1},
        Robinson Merillat\IEEEauthorrefmark{2}, and
        Sam Sartor\IEEEauthorrefmark{3}
    }
    \IEEEauthorblockA{%
        Department of Computer Science,
        Colorado School of Mines\\
        Golden, Colorado\\
        Email:
            \IEEEauthorrefmark{1}jonathanevans@mines.edu,
            \IEEEauthorrefmark{2}rdmerillat@mines.edu,
            \IEEEauthorrefmark{3}ssartor@mines.edu
    }
}

\begin{document}

\maketitle

\begin{abstract}
    Virtual Reality (VR) Technology has been developing rapidly over the past
    decade. Current solutions such as Unity attempt to use old programming
    languages and paradigms to implement VR environments and thus limit
    developers' abilities to create new and unique environments. With every
    cutting edge technology, new paradigms and design patterns must be invented.
    In this paper, we discuss a new deferred immediate mode (DIM) application
    architecture suitable for implementing large virtual reality applications.
    We present a library which implements this architecture and a few case
    studies of this library in use.
\end{abstract}

\section{Introduction}
The Virtual Reality (VR) market is growing rapidly. The International Data
Corporation (IDC) projects that revenues for the combined Augmented Reality (AR)
and Virtual Reality markets will grow from \$5.2 billion in 2016 to more than
\$162 billion in 2020~\cite{IDC:2016:VR-industry}. This flourishing new industry
has created an exciting  new field of Software Engineering with great potential
for revolutionary new design paradigms and program architectures.

Most current frameworks and libraries attempt to apply old design paradigms and
program architectures that are well-suited for 2D user interfaces and rendering
3D environments to a flat screen to virtual reality. Although these endeavors
have been successful, they do not fully explore potential new design paradigms
and program architectures.

Our goal was to develop innovative environments with % TODO: something here
We wanted to use a modern, fast, and usable framework to create these virtual
reality applications. After analyzing the current solutions, realized that the
current methods had many drawbacks. We will examine them in turn.

\subsection{Frame Rate}
Virtual Reality requires a very high frame rate. A virtual reality program must
be capable of generating 90 frames per second (fps). Because there are two eyes,
and thus two screens to render to, the effective required frame rate is 180fps.
Achieving this frame rate is resource intensive and requires highly efficient
and optimized code. Additionally, multithreading is imperative so that
long-running processes can occur without blocking the user interface. This is
unlike a traditional desktop user interface where blocking the UI process for a
second does not have a major affect on the usability of the program.

\subsection{3D Environment}
% TODO: not just a projection onto 2D screen

\subsection{Programmability}
% TODO:

\subsection{Open Source}
% TODO: because it should be

% TODO: Some sort of transition from talking about the drawbacks to this
We chose to take a step back from the predominant design philosophies and
explore alternative architectures.

In this paper, we present a new program architecture and user interface design
paradigm which builds on and combines concepts from many previous architectures.

\section{Evolution of Deferred Immediate Mode}
% TODO: Give brief overview of our "adventures in VR library design"

\subsection{Entity Component System}
% TODO:

\subsection{React-Like Architecture}
% TODO:
% TODO: mention A-Frame

\subsection{Event Tree}

\section{Deferred Immediate Mode}
% TODO: describe general ideals of our architecture (programming language and
% implementation agnostic)

\subsection{Deferred}
% TODO: describe why this is cool

\subsection{Immediate Mode}
% TODO: describe why this is cool

\section{Flight}
% TODO: this section should describe the implementation of DIM (or whatever we
% call it)
Flight is our implementation of a VR UI library using the DIM architecture. It
is written in Rust and is designed to be highly modular. It implements the DIM
architecture using Rust \texttt{FnOnce} closures and resolves state using a
\textit{guru} system.

\subsection{Language}
% TODO: why Rust? Type system.

\subsection{Dependencies}
% TODO: WebVR, nalgebra, gfx

\subsection{Modular}
% TODO:

\subsection{Deferred}
% TODO: describe how we use FnOnces

\subsection{Immediate Mode}
% TODO: describe how we did this (update function)

\subsection{Gurus} % TODO: maybe lump this in with something else?
% TODO: describe how the gurus help resolve things

% TODO: add subsubsections about the various gurus?

\section{Case Studies}
% TODO: Explain that we were creating these applications for a VR independent
% study.

\subsection{Let's Get Physical and Snowflakes --- Physics}
% TODO: Describe how physics, spawning, etc. works in Immediate Mode & Physics
% guru

\subsection{Workbench --- SOMETHING} % TODO
% TODO: Describe how it is used to resolve what is being pointed at.

\subsection{{\name} --- Global User Interface}
% TODO: Describe how we are using it to make the torus thing

\section{Comparison to Alternative Libraries}
% TODO: Overview of the types of alternatives

\subsection{A-Frame}
% TODO: Describe why it sucks compared to Flight

\section{Conclusion}
% TODO: Conclude (duh)

{\printbibliography}

\end{document}
