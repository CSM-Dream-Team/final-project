\documentclass[titlepage,12pt]{article}
\usepackage[margin=1in]{geometry}
\usepackage{enumitem}
\usepackage[shortlabels]{enumerate}
\usepackage{titlesec}
\usepackage{textcomp}
\usepackage{amsmath}
\usepackage[makeroom]{cancel}

\title{Final Project Proposal}
\author{Sumner Evans, Robbie Merillat, Sam Sartor}
\date{\today}

\begin{document}
\maketitle

\section{Introduction}

VR Technology has been developing rapidly in the 21st century. Current solutions
such as Unity attempt to use old programming languages and paradigms to
implement VR environments and thus limit developers' abilities to create new and
unique environments. With every cutting edge technology, new paradigms and
design patterns must be invented. This project intends to explore and implement
these paradigms and design patterns.

\section{Objectives}

The goal of this project is to experiment with a variety of paradigms and design
patterns to see which work best for implementing VR environments. To do this, we
will implement four different VR environments where users can move around and
interact with a variety of virtual objects. Each team member will build one
environment individually and the fourth will be built by the whole team and will
connect the individually-built environments.

\subsection{Environments and Interactions}
The following sections describe the concepts for our four environments and the
interactions that can occur in each environment.

\subsubsection{Sumner --- Snowflakes}
Snowflakes will be a winter-themed environment where users can create their own
structures by manipulating snow blocks. The environment will have ``snow'' on
the ground and winter themed items in the environment such as snowmen. The user
will be able to create new snow blocks by pointing both controllers at the
ground and pulling both of the triggers. This action will create a snow block
being held between the users hands. When the user releases the triggers, the
snow block will fall in a physically-accurate manner with collision detection.

\subsubsection{Robbie --- Let's Get Physical}
An environment where you can throw lots of objects around and see them interact
in a physically-accurate manner.

\subsubsection{Sam --- Workbench}


\subsubsection{Team --- VRsh}
A VR shell where users can interact with a variety of widgets as well as open
other programs (environments).

More than just widgets, vrsh is a graphical way
of sending commands. My thought is that we actually create some sort of
templeting engine, that allows us to wrap traditional, UNIX-y commands with vrsh
widget systems.

\section{Plan of Action}

This project will have four main stages that correlate with the due dates for
the individual assignments and the final project milestones.

\begin{enumerate}[leftmargin=*]
    \item [10/18] \textbf{Individual Assignments} --- Demo the individual
        environments
    \item [11/03] \textbf{Milestone I} --- Building the VRsh environment
    \item [11/17] \textbf{Milestone II} --- Integrating the VRsh and individual
        environments
    \item [12/08] \textbf{Final Code Submission} --- Final touches to the VRsh
        environment.
\end{enumerate}

In addition to the above code submissions, we will also produce a final report
along with the Final Code Submission describing the work that we did and the
lessons we learned while implementing the project.

\subsection{C-MAPP Event Readiness}
Our goal is to have this project ready for the C-MAPP event in January. To do
this, we must have a finished product and report by the end of this semester.

\section{References and Dependencies}

Because of the performance requirements and complexity of virtual reality
software, we decided to use the Rust programming language for all the projects
in this independent study.

Our individual and final projects will all be built using our own shared,
open-source library called \texttt{flight}, which will be continually updated to
reflect what we learn while developing these projects. As of writing,
\texttt{flight} incorporates our own VR hardware interface, several custom
rendering environments, mesh manipulation tools, and a few asset loading
utilities.

Although most of the planned features of our shell can be implemented using the
Rust standard libraries, there are several other tools that we will be depending
on.  The most notable are \texttt{gfx} (a GPU graphics wrapper),
\texttt{nalgebra} (vector and matrix math),\ \texttt{ncollide} (geometric
operations such as collisions and intersections), \texttt{nphysics} (physics
engine), OpenVR (Valve's VR hardware SDK), image (raster image loader), and
futures (asynchronous events). With the exception of OpenVR, all these libraries
can be found on \texttt{crates.io}.

\end{document}
